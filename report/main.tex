%%%%%%%%%%%%%%%%%%%%%%%%%%%%%%%%%%%%%%%%%%%%%%%%%%%%%%%%%%%%%%%%%%%%%%%%%%%%%%%%
%\documentclass[letterpaper, 10 pt, conference]{ieeeconf}  % Comment this line out
                                                          % if you need a4paper
\documentclass[a4paper, 10pt, onecolumn]{ieeeconf}      % Use this line for a4
                                                          % paper

%\IEEEoverridecommandlockouts                              % This command is only
                                                          % needed if you want to
                                                          % use the \thanks command
%\overrideIEEEmargins
% See the \addtolength command later in the file to balance the column lengths
% on the last page of the document



% The following packages can be found on http:\\www.ctan.org
\usepackage{graphics} % for pdf, bitmapped graphics files
\usepackage{epsfig} % for postscript graphics files
\usepackage{subfigure}
%\usepackage{mathptmx} % assumes new font selection scheme installed
%\usepackage{times} % assumes new font selection scheme installed
\usepackage{amsmath} % assumes amsmath package installed
\usepackage{amssymb}  % assumes amsmath package installed
\usepackage{booktabs}
\usepackage{float}
\usepackage{amsmath}
\usepackage{setspace}
\usepackage{verbatim} %for multiline comment
\usepackage{amsfonts}
\usepackage{amssymb}
%\usepackage[usenames,dvips,pdftex]{color}
%\usepackage[pdftex]{color,graphicx}
%\usepackage{sty/algorithm}
\usepackage{setspace}
\usepackage[mathscr]{euscript}
%\usepackage{subfig}
\usepackage[margin=1in]{geometry}
%\RequirePackage{setspace}
\usepackage{mathptmx}
\graphicspath{{figs/}}
\usepackage{amsmath,amssymb,amsfonts,bbm,stmaryrd}
\usepackage{algorithm,algorithmicx,listings}  % algorithms
\usepackage[noend]{algpseudocode}	
\usepackage[mathscr]{euscript}
% necessary for algorithmicx
%=================================================================%
% Commands
\def\liminf{\mathop{\lim\,\inf}\limits}	% EXAMPLE: \liminf_n A_n
\def\limsup{\mathop{\lim\,\sup}\limits}	%
\def\argmin{\mathop{\arg\,\min}\limits}	%
\def\argmax{\mathop{\arg\,\max}\limits}	%
\newcommand{\indicator}{\mathbbm{1}}
\newcommand{\txbx}[1]{\boxed{\text{#1}}}

\newtheorem{lemma}{Lemma}
\newtheorem{theorem}{Theorem}
\newtheorem{proposition}{Proposition}
\newtheorem{definition}{Definition}
\newtheorem{assumption}{Assumption}
\newtheorem{remark}{Remark}
\newtheorem{problem}{Problem}

\title{\LARGE \bf
Learning and Optimization with Submodular Functions
}
\author{Bharath Sankaran}

\begin{document}
\maketitle
\thispagestyle{empty}
\pagestyle{empty}
%%%%%%%%%%%%%%%%%%%%%%%%%%%%%%%%%%%%%%%%%%%%%%%%%%%%%%%%%%%%%%%%%%%%%%%%%%%%%%%%
\section{Motivation}
In many naturally occurring optimization problems one needs to ensure that the definition of the optimization problem lends itself to solutions that are tractable to compute. In cases where exact solutions cannot be computed tractably, it would be beneficial to have strong guarantees on the tractable approximate solutions. In order operate under these criterion most optimization problems are cast under the umbrella of convexity or submodularity. In this report we will study the design and optimization over a common class of functions called submodular functions. \\

Submodular functions characterize a wide variety of naturally occuring optimization problems and the property of submodularity of set functions have deep theoretical consequences with wide ranging applications. Informally the property of submodularity of set functions can be characterized with the intuitive notion of diminishing returns. This property states that adding an element to a smaller set has more value than adding it to a larger set. Common examples of submodular functions which are monotone are entropies, concave functions of cardinality and matroid rank functions and examaples of non-monotone are graph cuts, network flows and mutual information. \\

In this paper we will review the formal definition of submodularity; the optimization of submodular functions, both maximization and minimization; and finally discuss some applications in relation to learning using submodular functions.


\section{What is Submodularity: Formal Definition}
Formally we can define submodularity as the property of set functions $f:2^V \rightarrow \Re$, which assign each subset $S \subseteq V$ a value $f(S)$. Here $V$ is a finite set called the {\it ground set}. We also assume that $f(\emptyset) = 0$.\\

{\bf Definition 1}: {\it A set function $f:2^V \rightarrow \Re$ is called submodular if it satisfies

\[
 f(X) + f(Y) \geq f(X \cup U) + f(X \cap Y) \text{  } \forall \text{  } X,Y \subseteq V
\]

In the above definition the function $f$ lends itself to different forms in different application domains. For instance in a Machine Learning context $f$ could be a function that evaluates information of a given set, i.e entropy. Using this notion, we can easily introduce the property of diminishing returns by using an equivalent definition for submodularity.}\\

{\bf Definition 2}: {\it A set function $f:2^V \rightarrow \Re$ is called submodular if it satisfies

\[
 X \rightarrow f(X\cup {k}) - f(X) \text{ is non-increasing }
 \]
 \[
 f(X \cup {k}) - f(X) \geq f(Y \cup {k}) + f(Y) \text{  } \forall \text{  } X\subset Y \text{  } \forall k \notin X
\]

Finally, a set function $f$ is called {\bf supermodular} if $-f$ is submodular, and if $f$ is both sub and supermodular then the function is called a {\bf modular function}.}

\subsection{Notation I}
\label{sec:notation}
In this section we will introduce some notation that we will consistently maintain through the course of this document, unless specified. The ground set over which the submodular functions are defined will be denoted by $V$ with cardinality $n$. For a vector $x \in \Re^V$ and a subset $Y \subseteq V$ we define $x(Y) = \underset{u \in Y}{\operatorname{\sum }}\text{ } x(u)$. We can naturally extend this definition to capture the positve and negative parts of the vector $x$ as $x^+\in \Re^V$ and $x^-\in \Re^V$, where $x^+(u) = max\{x(u),0\}$ and $x^-(u) = min\{x(u),0\}$. For a submodular function $f$ we define a polyhedral convex set $P(f)$ called the {\it submodular polyhedron}:
\[
 P(f) = \{ x\in \Re^V \mid x(S) \leq f(S) \text{ }\forall S\subseteq V \}
\]
The face of $P(f)$ for which $x(V) = f(V)$ which defines the {\it base polyhedron}:
\[
 B(f) = \{ x \in P(f) \mid x(V) = f(V)\}
\]
The elements of $B(f)$ as bases of the set $V$ or the polyhedron $P(V)$.


\subsection{Properties of Submodular Functions}
The basic properties of submodular functions are enumerated below. These properties will help us recast many of our optimization objectives as submodular optimization problems.\\

\begin{itemize}
 \item {\bf Lemma 1.1: \it Closedness Properties}: Submodular functions are closed under non negative linear combinations, i.e if $\{f_1,f_2,...,f_k\}$ are submodular then the function $g(X) = \sum\limits_{i=1}^k\text{ } \alpha_if_i(X) \text{ is submodular }\forall \alpha_i \geq 0$.\\
 
 {\bf Corollary}: The sum of a modular and submodular function is a submodular function.\\
 
 {\bf Corollary [Restriction/marginalization]}: if $Y\subset V$, then $X\rightarrow f(X \cap Y)$ is submodular on $V$ and $Y$.\\
 
 {\bf Corollary [Contraction/conditioning]}: If $X \subseteq Y$ and $f$ is submodular, then $g(X) = f(Y\setminus X)$ is submodular. Equivalently if $Y\subset V$, then $X\rightarrow f(X \cup Y) - f(Y)$ is submodular on $V$ and $V\setminus Y$\\
 
 \item {\bf Lemma 1.2: Partial Minimization}: Monotone submodular functions remain submodular under truncation, i.e if $f(X)$ is submodular then $g(X) := min\{f(X),c\}$ for any constant $c$ is submodular.\\
 
 {\bf Note:} This property is not necessarily preserved for max or min for two submodular functions.\\
 
 \item{\bf Lemma 1.3: Cardinality Based Functions} If $f(X)$ is a submodular function, then $g(X) = \phi(f(X))$ is also submodular if $\phi()$ is a concave function.\\
 
 \item{\bf Lemma 1.4: Lov$\acute{\bf a}$sz Extension} A function $f$ is submodular function, iff its Lov$\acute{a}$sz Extension $\hat{f}$ is convex, where
 \[
  \hat{f}(c) = max\{c^Tx \mid x(U) \leq f(U) \text{ } \forall \text{ } U \subseteq V \text{ and } c\in[0,1]^n\}
 \]

\end{itemize}

\section{Submodular Optimization}
Submodular functions have many interesting connections with convex and concave functions as demonstrated by {\bf Lemma 3}. Just as minimization of convex functions can be done efficiently, unconstrained submodular minimization is also possible in strongly polynomial time. Submodular function maximization in contrast is a NP hard combinatorial optimization problem, but approximate solutions can be found with guarantees. In fact a simple greedy solution method obtains a $(1 - 1/e)$ approximation, given that we are maximizing a non-decreasing submodular function under matroid constraints.

\subsection{Submodular Function Minimization}
Submodular function minimization can be divided into two categories, exact and approximate algorithms. Exact algorithms obtain global minimizers for a problem whereas approximate algorithms only achieve an approximate solution i.e for a set $X$ the solution $f(X) - \underset{Y \subset V}{\operatorname{min }}\text{ } f(Y) \leq \epsilon $, where $\epsilon$ is as small as possible. If $\epsilon$ is less than the minimum absolute difference between non equal values of $f$, the solution computed corresponds to the exact solution.

An important practical aspect of submodular function minimization is that most algorithms come with online approximation guarantees {\it due to a duality relationship}, which we will detail in the following subsections.\\

\subsubsection{\bf Submodular Function Minimizers}
For the lemmas stated below, we consider $f$ to be a submodular function where $\{f:2^V \rightarrow \mathds{R} \mid f(\emptyset) = 0\}$.\\

\begin{itemize}
 \item {\bf \lemma (Lattice of minimizers for submodular functions)}: The set of minimizers of $f$ is a lattice, i.e if $X$ and $Y$ are minimizers then $X\cup Y$ and $X \cap Y$ are also minimizers. This is evident from {\bf Definition 1} and {\bf Lemma 1.1}\\

 \item {\bf \lemma (Diminishing return property of minimizers of submodular functions)}: The set $X \subset V$ is a minimizer of $f$ on $2^V$ iff $X$ is a minimizer of $2^X \rightarrow \mathds{R}$ defined as $Y \subset X \rightarrow f(Y)$ and if $\emptyset$ is a minimizer of the function from $2^{V\setminus X} \rightarrow \mathds{R}$  then it is defined as $Y \subset V\setminus X \rightarrow f(Y \cup X) - f(X)$ . This can be easily shown from {\bf Definition 1}.\\
 
  {\it Corollary 6.1 : (Norm Characterization)}: Suppose $\hat{x}$ is a minimizer of 
 \[
  \underset{x}{\operatorname{min }}\text{ } \| x\|_2^2 \text{ subject to } x\in B(f)
 \]
 Then a minimizer $A$ for $f$ can be obtained as follows:
 \[
   A = \{ u \in V \mid \hat{x}(u) \leq 0\}
 \]

  \item{\bf \lemma (Dual of minimization of submodular functions)}: \[ \underset{X \subset V}{\operatorname{min }}\text{ } f(X) =  \underset{x \in B(f) }{\operatorname{max }}\text{ } x^-(V) = f(V) - \underset{x \in B(f) }{\operatorname{min }}\text{ } \|x\| \]
  As mentioned in Section \ref{sec:notation} $(x^-)_k = \min\{x_k,0\} \text{ } \forall k \in V$. If $X \subset V$ and $x \in B(f)$, we have $f(X) \geq x^-(V)$ with equality iff $\{ x < 0 \} \subset X \subset \{ x\leq 0\}$ and $x(X) = f(X)$. \\
  \[
   \underset{X \subset V}{\operatorname{min }}\text{ } f(X) =  \underset{x \in P(f), x\leq 0 }{\operatorname{max }}\text{ } x(V)
  \]
  Again if $X\subset V$ and $x\in P(f) \mid x \leq 0$, then $f(X) \geq x(V)$ iff $\{x \leq 0 \} \subset X$ and $s(X) = f(x)$.\\

\end{itemize}
\subsubsection{Minimum Norm Point Algorithm}
As an example of Submodular function minimization we present the minimum norm point algorithm. A non combinatorial approach proposed by Fujishige \cite{Fujishige} is based on the norm characterization of the minima of $f$ shown in {\bf Lemma 2.2}. Fujishige uses Wolfe's algorithm \cite{Wolfe} which was developed to minimize the $L_2$ norm of a vector in a convex hull of a finite set of points $P \in \mathds{R}^n$. This method maintains the vector $x$ as a convex combination of points $S$ and iterates over the following steps:
\begin{enumerate}
\item A new point from $P$ with a norm with respect to $x$ is added to to set $S$.
\item A point with the minimum norm $x$ is computed in the affine hull of $S$.
\item The minimum norm point $x$ is projected onto the convex hull of $S$. 
\end{enumerate}

In the case of submodular functions, one needs to search through the set of all bases $P$ which is exponential in size. This issue is circumvented by using Edmonds Greedy Algorithm \cite{Edmond}.

\begin{algorithm}[htb]
\caption{Minimum Norm Point Algorithm}
\label{alg:min_norm_point}
\begin{algorithmic}[1]
\footnotesize
\State {\bf Initialization}: x $\leftarrow$ extreme base generated using arbitrary ordering, $S \leftarrow \{x\}$
\Loop
  \State {\bf Selection of new base using Edmonds Greedy Algorithm}: $y' \leftarrow  \underset{y}{\operatorname{argmin }}\text{ } x^Ty \text{ } \forall y\in B(f)$
\If{$x^Ty' = x^Tx$}
\State {\bf return} x
\Else
\State $S \leftarrow S \cup \{y'\}$
\EndIf
\State {\bf Minimization over affine hull of $S$}: $z \leftarrow  \underset{y}{\operatorname{argmin }}\text{ } {\|y\|_2^2 \text{ where } y\in S}$
\State {\bf Projection on to convex hull of $S$}: 
\While{ $z \notin \mathrm{relint}(\mathrm{conv}(S))$}
               \State $z \leftarrow \text{ intersection of } [z,x] \text{ and } S$
		\State $S \leftarrow \text{ face of $S$ intersected by } [z,x]$
\EndWhile
\State $x\leftarrow Z$
\EndLoop
\end{algorithmic}
\end{algorithm}

\subsection{Submodular Function Maximization}
Problems for the form $\underset{X\subset V}{\operatorname{max }}\text{ } f(X)$ for any submodular function $f$ occurs in various applications. Problems of these kind are known to be NP-Hard. Feige and Mirrokni \cite{Feige07maximizingnon-monotone} showed that maximizing for non-negative submodular functions a random subset achieves at least 1/$4^{\text{th}}$ the optimal value and local search techniques achieve at least a 1/2. Though these problems are NP-Hard, a $(1 - 1/e)$ approximation can be obtained when maximizing a non-decreasing submodular function under matroid constraints.
The solution to the arbritary matroid constraint was shown more recently by Vondrak et al \cite{Vondrak}. The initial result of an $(1 - 1/e)$ approximation was shown by Nemhauser  \cite{Nemhauser} in the 70s, but the result was only applicable to uniform matroid (cardinality) constraints. The solution to the uniform matroid contraint consists of a simple greedy algorithm that has implication in online learning and adaptive submodularity.
\begin{itemize}
 \item {\bf \lemma Local minima for submodular function minimization:} \\ Given a submodular function $\{f:2^V \rightarrow \mathds{R} \mid f(\emptyset) = 0\}$ and  $X \subset V$ such that $\forall k \in X$, $\text{ }f(X\setminus \{k\}) \leq f(X)$  and $\forall k \in V\setminus X, \text{ } f(X\cup \{k\}) \leq f(X)$ holds true.
 \[ \text{Then, } \forall Y\subset X \text{ and } \forall Y\supset X \text{ , }f(Y) \leq f(X) \] 
\end{itemize}
\subsubsection{Greedy Algorithm for Monotone Submodular Function Maximization with Uniform Matroid Constraints}
Maximization for arbitary constraints can be achieved using Vondrak's algorithm. In this document we'll focus on the greedy algorithm as it has implications in the online learning domain. {\bf Note:} Maximization can also be formulated using the base polyhedron given we have $f$ and it's lov$\acute{a}$sz extension $\hat{f}$. In this case maximization is equivalent to finding the maximum l1-norm point in the base polyhedron. See \cite{Bach} for more details.
For monotone submodular maximization subject to uniform matroid constraints, we need to find a set $X^*\subseteq V$ such that
\[
 X^* = \underset{\|X\| \leq n}{\operatorname{argmax }}\text{ }f(X)
\]

where $n$ is the cardinality (uniform matroid) constraint. Though this problem is NP-hard we can get an approximate solution with an approximation of $(1-1/e)$ of the optimal solution.  The algorithm for obtaining this solution is shown in Algorithm \ref{alg:greedy}.
\begin{algorithm}[htb]
\caption{Greedy Algorithm}
\label{alg:greedy}
\begin{algorithmic}[1]
\footnotesize
\State {\bf Initialization}: Start with $X = \emptyset$
\For{ i = 1 to n }
  \State $y':= \underset{y}{\operatorname{argmax }}\text{ }f(X\cup {y}) $
  \State $X := X \cup {y'}$
\EndFor
\end{algorithmic}
\end{algorithm}


\subsection{Adaptive Submodularity}
The process of adaptively making decision with uncertain outcomes is fundamental to many problems with partial observability. In such situations decision maker needs to make a sequence of decisions taking by accounting for past observations and adapting accordingly. It has been shown by Golovin and Krause \cite{Golovin} that if a problem is adaptively submodular, then an adaptive greedy algorithm is guaranteed to obtain near optimal solutions.
For the notion of clarity in the discussion of Application domains, we introduce the notion of adaptive submodularity and adaptive monotonicity in this section.

\subsubsection{Preliminaries and Notation II}
Let $V$ be the ground set. Assuming each item of the set $x\in V$ can take a number of states from a set of possible states $O$, we represent item states as $\phi:V \rightarrow O$ which is a function that gives the realization of the states of all items in the ground set. Hence $\phi(x)$ is the state of $x$ under the realization $\phi$. Now consider a random realization characterized by the random variable $\Phi$. Then we can assume a prior probability distribution over realizations as $p(\phi) := \mathbb{P}[\Phi = \phi]$. 
In cases where we observe only one realization $\Phi(x)$ at a time, as we pick an item $x \in V$ at a time; we can represent our observations so far with a partial realization $\chi$, i.e a function of a subset of $V$ and its observed states. Hence $\chi \subseteq V \times O$ is $\{(x,o):\chi(x) = o\}$. Here we denote the domain of $\chi$, i.e the set of items observed in $\chi$ as $dom(\chi) = \{ x: \exists o.(x,o) \in \chi \}$. When a partial realization $\chi$ is equal everywhere with $\phi$ in the $dom(\chi)$, they are {\it consistent} $\phi \sim \chi$.
This implies that all the items observed with specific states in $\chi$ have also been observed with the same states in $\phi$. Now we extend this notion to subsets by saying if $\chi$ and $\chi'$ are consistent with $\phi$ and $dom(\chi)\subseteq dom(\chi')$, then $\chi$ is a subrealization of $\chi'$. In a Partially Observable Markov Decision Problem (POMDP), sense partial realization are our belief states. They determine our posterior belief given the effect of all our actions and observations.
\[
 p(\phi\mid\chi) := \mathbb{P}[\Phi=\phi | \Phi \sim \chi]
\]

{\bf \definition (Conditional Expected Marginal Benefit):} 
{\it Given a partial realization $\chi$ and a item $x$, the conditional expected marginal belief of $x$ conditioned on having already observed $\chi$ is denoted by $\delta(x\mid\chi)$
\[
 \delta(x\mid\chi) = \mathbb{E}[f(dom(\chi)\cup\{x\},\Phi) - f(dom(\chi),\Phi) \text{ } \mid \text{ } \Phi \sim \chi]
\]
}
{\bf \definition (Adaptive Monotonicity):}
{\it A function $f:2^V \times O^V \rightarrow \Re_+$ is adaptive monotone with respect to the distribution $p(\phi)$ if the conditional expected marginal benefit of any item $x$ is non-negative, i.e $\forall \chi$ with $\mathbb{P}[\Phi \sim \chi] \geq 0$ and all $x\in V$ }
\[
 \delta(x\mid\chi) \geq 0
\]

{\bf \definition (Adaptive Submodularity):}
{\it A function $f:2^V \times O^V \rightarrow \Re_+$ is adaptive submodular with respect to the distribution $p(\phi)$ if the conditional expected marginal benefit of any fixed item does not increase as more items are selected and their states are observed, i.e if $f$ is adaptively submodular w.r.t to $p(\phi)$ if $\forall \text{ } \chi$ and $\chi'$ where $\chi$ is a subrealization of $\chi'$ and all $x\in V \setminus dom(\chi')$ , the following condition holds true:}
\[
 \delta(x\mid\chi) \geq \delta(x\mid\chi')
\]
Given these definitions we can now use the greedy algorithm defined in Algorithm \ref{alg:greedy_adap} to give an $\alpha$ approximation to the best greedy solution for online maximization problems of adaptively montone submodular functions. This means we find an $x'$ such that
\[
 \delta(x'\mid\chi) \geq \frac{1}{\alpha}\delta(x\mid\chi)
\]

The budget for these maximization problems OR the number of rounds we'd like to maximize is similar to the cardinality constraint of submodular problems.

\begin{algorithm}[htb]
\caption{$\alpha$-Approximate Greedy Adaptive Algorithm}
\label{alg:greedy_adap}
\begin{algorithmic}[1]
\footnotesize
\State {\bf Input}: Budget $n$, ground set $V$, $p(\phi)$ and function $f$ 
\State {\bf Output}: $X \subset V$ where $\|X\| = n$
\State {\bf Initialize:} $X \leftarrow \emptyset$ and $\chi \leftarrow \emptyset$
\For{ i = 1 to n }
  \State $\forall x \in V\setminus X \text{; Evaluate } \delta(x\mid\chi) = \mathbb{E}[f(dom(\chi)\cup\{x\},\Phi) - f(dom(\chi),\Phi) \text{ } \mid \text{ } \Phi \sim \chi]$
  \State $x^* = \underset{x}{\operatorname{argmax }}\text{ } \delta(x\mid\chi)$
  \State $X \leftarrow X \cup {x^*}$
  \State $\text{Observe :} \Phi(x^*) \text{; Update: } \chi \leftarrow \chi \cup {x^*,\Phi(x^*)}$
\EndFor
\end{algorithmic}
\end{algorithm}

\section{Applications}
\subsection{Feature Selection: Marjan}
In machine learning and statistics, feature selection is one of the most important concepts. The aim of this process is to select a subset of relevant features for use in the model construction.  In real world problems we usually encounter a large number of features for data points which are mainly redundant or irrelevant. These features might cause problems like complexity  and overfitting of model to the training data. By omitting the redundant and irrelevant features, we can gain:
\begin{itemize}
\item improved model interpretability
\item shorter training and testing time
\item enhanced generalization of the model by reducing overfitting
\end{itemize} 

In feature selection process, we  search among features and choose the ones that are more informative in our problem.This definition  can be interpreted as an optimization problem for choosing a subset of features which maximize the mutual information between features and labeling function.

Hence, if $S$ indicates the set of all features and $s$ indicates the chosen feature set, and assuming that $||s||_1\leq b$, we can write the problem as:
\begin{equation*}
max_s I(y;x_s)
\end{equation*} 
where $y$ is the labeling function.

\subsubsection{Submodularity}

Suppose $A \subset B \subset S$ and $m \not \in B$, for proving the submodularity, we should prove

\begin{eqnarray}
I(y;x_A \cup x_m) - I(y;x_A) &\geq & I(y;x_B \cup x_m) - I(y;x_B) \nonumber \\
\Leftrightarrow H(y|x_A)-H(y|x_A,x_m) &\geq & H(y|x_B)-H(y|x_B,x_m) \label{f1}
\end{eqnarray}
  
 we can write
\begin{eqnarray} 
	&&H(y|x_A)-H(y|x_A,x_m) \nonumber \\
	&=&H(y|x_A)+ H(x_m|x_A)-H(y,x_m|x_A)\nonumber \\
	&=&H(y|x_A)+ H(x_m|x_A) -H(y|x_A) - H(x_m|x_A,y)\nonumber \\
	&=& H(x_m|x_A)- H(x_m|x_A,y) \label{f2}
\end{eqnarray} 

 Substituting this term into equation \ref{f1}, it can be seen that there are cases in which the function is not submodular.
\begin{itemize}
\item {\bf \lemma} If $x_i$s are all conditionally independent given y, then the function is submodular \cite{krausefeature}.
\end{itemize}

This constraint is met in many practical problems in machine learning area. If $x_i$s are all conditionally independent given y, then equation \ref{f2} can be written as,
\begin{equation*}
H(y|x_A)-H(y|x_A,x_m)=H(x_m|x_A)-H(x_m|x_A,y)
\end{equation*}
and if we substitute this in equation \ref{f1},


\begin{eqnarray*}
I(y;x_A \cup x_m) - I(y;x_A) &\geq & I(y;x_B \cup x_m) - I(y;x_B) \nonumber \\
\Leftrightarrow H(x_m|x_A) &\geq & H(x_m|x_B)
\end{eqnarray*} 

Hence, the problem of feature selection can be written as a maximization of a submodular function.\cite{jie}


\subsection{MAP Inference: Bharath}
In this section we will specifically look at the problem of Maximum a Posteriori Inference on graphs. To analyze the algorithms in greater detail, we would like to introduce a few preliminary notions, including the concept of Polymatroids. 
\subsubsection{Polymatroids}
The notion of submodularity was first studied in the context of matroids. A set system $(V,\mathcal{F})$ is defined by a ground set $V$ and a family of subsets $\mathcal{F} \subseteq 2^V$. Such a system is a matroid if
\begin{itemize}

\item $\emptyset \in \mathcal{F}$
\item $\text{ if } X \subseteq Y \in \mathcal{F} \text{ then } X\in \mathcal{F}$
\item $\text{ if } X,Y \in \mathcal{F} \text{ and } \|X\| > \|Y\| \text{ } \exists e\in X\setminus Y \text{ such that } Y+e\in\mathcal{F}$
\end{itemize}

Now we define a function $\rho$ called a rank function, which assigns a natural number to each subset of $V$. This rank function is analagous to rank functions of matrices, in fact the matroid a which is the set of linearly independent columns of a matrix $A$ is called a metric matroid. We define our matroid {\it rank function} $\rho$ as follows
\[
\rho(X) = max\{\|F\| \text{ }\mid \text{ } F\in\mathcal{F}, \text{ }F\subseteq X\}
\]
If $\rho$ is a rank function of a matroid $(V,\mathcal{F})$ then the following properties hold:
\begin{itemize}
\item $\rho(X) \leq \|X\| \text{ }\forall X\subseteq Y$
\item $\rho \text{ is non decreasing: if } X \subseteq Y \subseteq V \text{ then } \rho(X) \leq \rho(Y)$
\item $\rho \text{ is submodular }$
\end{itemize}

If a set function $\rho$ satisfies the above properties for a ground set $V$ then the resulting structure $(V,\rho)$ is called a {\it polymatroid}. Similarly if $(V,\rho)$ is a polymatroid, then the family of subsets
\[
\mathcal{F} = \{ F\subseteq V \text{ }\mid \text{ } \rho(F) = \|F\|
\]
defines a matroid $(V,\mathcal{F})$.

\subsubsection{Cuts in Graphs, Energy Minimization and MAP Inference}
Consider a directed graph $G = (V,A,W)$ with positive edge weights $w:A\rightarrow\mathds{R}^+$. We can define a {\bf positive directed cut} for a given set of vertices $S\subseteq V$ as the set of edges starting in $S$ and ending in $V\setminus S$ $:\delta^+(S) = \{(i,j)\in A\text{ }\mid \text{ }i \in S, j\in V\setminus S\}$, Similarly a negative directed cut is $\delta^-(S) = \{(i,j)\in A\text{ }\mid \text{ }i \in S, j\in V\setminus S\}$. Finally the cut $\delta(S) = \delta^+\cup\delta^-$, this for an undirected graph would be the set of edges with exactly one end in $S$. Hence we can define the weight of a cut as
\[
 f^+ = \underset{e\in\delta^+(S)}{\operatorname{\sum}} w(e), \text{ }  f^- = \underset{e\in\delta^-(S)}{\operatorname{\sum}} w(e), \text{ } f = \underset{e\in\delta^+(S)}{\operatorname{\sum}} w(e)
\]

Given these cut functions one can note that these cut functions are submodular.

\begin{itemize}
\item {\bf \lemma} {\it The cut functions $f^+$, $f^-$ and $f$ are submodular}\\
{\it Proof :} For the function $f$, suppose $X,Y\subset V$ then,
\[
f(X) + f(Y) - f(X\cup Y) - f(X\cap Y) = \underset{i\in\{X\setminus Y\}, j\in\{Y\setminus X\}}{\operatorname{\sum}} w(i,j) + \underset{i\in\{X\setminus Y\}, j\in\{Y\setminus X\}}{\operatorname{\sum}} w(j,i)
\]
from the non-negativity of edge weights we can quickly conclude the above function is submodular. Similarly submodularity can be proved for $f^+$ and $f^-$.
\end{itemize}
 Now in order to formalize our notion of Maximum a posteriori estimation as a submodular function minimization problem, we introduce the following notation. Consider the function $E:{0,1}^n\rightarrow\mathds{R}$ defined over binary variables $X={x_1,...,x_n}$. Such functions are called \textbf{regular functions} \cite{Kolmogorov04whatenergy}. We can define an equivalent set function $\hat{E}$
\[
\hat{E}(S) = E(x) \text{ where } x_i = 1 \text{ if and only if } i \in S
\]
We define the class $\mathcal{F}^2$ to be functions that can be written as a sum of functions of up to two binary variables at a time.
\[
E(x_1,....x_n) = \underset{i}{\operatorname{\sum}} E^i(x_i) + \underset{i<j}{\operatorname{\sum}} E^{i,j}(x_i,x_j)
\]
The regularity of the binary function from $\mathcal{F}^2$ translates to submodularity of the equivalent set function. 

Given an input set of nodes $\mathcal{P}$ in a graph $\mathcal{G}$ and a set of labels $\mathcal{L}$, the labeling $l$ (which is a mapping from $\mathcal{P}$ to $\mathcal{L}$) can be deduced by minimizing some energy function. In graph based energy minimization problems in computer vision and machine learning, the standard form of the energy function used is as follows
\[
E(l) = \underset{p\in\mathcal{P}}{\operatorname{\sum}} D_p(l_p) + \underset{p,q\in\mathcal{N}}{\operatorname{\sum}} V_{p,q}(l_p,l_q)
\]

where $\mathcal{N} \subset \mathcal{P}\times\mathcal{P}$ is a neighbourhood set of nodes. $D_p$ is a cost function derived from assigning label $l_p$ to node $p$. $V_{p,q}$ is the cost of assigning labels $l_p,l_q$ to adjacent nodes $p,q$. If V is a non-convex function of $\|l_p - l_q\|$ which accounts for border labeling, the energy function $E(l)$ is called a discontinuity preserving energy function. This label assignment problem is similar to the bears similarity to the graph cut problem. As the labeling function is submodular in the context of the regular functions defined earlier. Minimizing this energy function E is equivalent to finding the minimum cut of the graph $\mathcal{G}$. However one should note that the solution depends on the exact form of the function $V$ and it cannot be convex as it leads to oversmoothing of borders. If $V(l_p,l_q)=T[l_p\neq\l_q]$, where T is the indicator function. This smoothness term is called the Potts Model. The solution shown above can be readily extended to more that two labels, or beyond the binary problem. We use the binary problem to motivate the result. This result is widely used in computer vision in the domains of image segmentation , stereo correspondence and multi-camera image reconstruction. Another widely used application of this approach is to find the Maximum a posetiori estimate of a Markov Random Field \cite{MRFKohli}.

Consider a set of random variables $X = \{X_1,....,X_n\}$ defined on a set $S$ such that the variable $X_i$ can take the value $x_i$ from the set $\mathcal{L} = \{l_1,...l_n\}$. The $X$ can be defined as a Markov Random field with respect to the neighbourhood set $N = \{N_i \text{ } \mid \text{ } i\in$ iff, the positivity property $P(x) > 0$ and the Markovian property $P(x_i\mid x_{S\setminus {i}}) = P(x_i \mid x_{N_i}) \text{ } \forall i \in S$. Here $P(x) = P(X = x)$ and $P(x_i) = P(X_i = x_i)$ and finally $P(X_1 = x_1,...,X_n = x_n)  = (X = x) \text{ where } x = \{x_i \mid i\in S\}$ is a realization of the field. Given these definitions the MAP estimate of the MRF can be formulated as an energy minimization problem, where energy corresponding to a realization of x (configuration of the field) is given by the negative log likelihood of the joint posterior probability of the MRF
\[
\phi(x) = -logP(x\mid D)
\]
Hence the corresponding energy function for the Potts model becomes 
\[
E(x) =  \underset{i\in S}{\operatorname{\sum}} \left(\phi(D|x_i) + \underset{j\in \mathcal{N}_i}{\operatorname{\sum}}\psi(x_i,x_j) \right)
\]

where 
\[
\phi(D|x_i)  = -logP(i\in S)
\]
 and 
 \[
 \psi(x_i,x_j) = \left\{ 
  \begin{array}{l l}
    K_{ij} & \quad \text{if $x_i \neq x_j$}\\
    0 & \quad \text{if $x_i = x_j$}
  \end{array} \right.
 \]
 Here $K_{ij}$ is penalty some cost which makes $\psi(x_i,x_j)$ non convex.
 Hence it can be seen that {\it the energy minimization problem solved by min-cut, max flow which yields the minimum energy solution is equivalent to the maximium a posteriori solution of a Markov Random Field}.
 


\subsection{Active Learning: David}
\subsection{Supervised learning theory}

In classic supervised machine learning, the learning algorithm (or \textit{learner}) is given the task of finding a response function $f: \mathcal{X} \mapsto \mathcal{Y}$ that predicts as accurately as possible the output \textit{response} $Y \in \mathcal{Y}$ for a given input observation	 $X \in \mathcal{X}$. Responses take a variety of forms. In classification, this may be a label from a discrete set of choices $\mathcal{Y} = \{ 1, 2, \dots\}$, while in regression it may be continuous. One of the most common tasks is binary classification, in which $\mathcal{Y} = \pm1$. We have some unknown underlying distribution $\mathcal{D}$ over the space of observations and responses $\mathcal{X} \times \mathcal{Y}$, so that observation-response pairs are sampled according to $(X, Y) \thicksim \mathcal{D}$. The learner chooses from candidate functions or \textit{hypotheses} in a hypothesis space $\mathcal{H}$ with the goal of minimizing the expected error or \textit{risk} $\epsilon_\mathcal{D}(h) = \mathbf{E}_{(X,Y)\thicksim \mathcal{D}}[\mathrm{err}(h(X), Y)]$. In other words, the learner's goal is to find $h^\ast$ that minimizes the risk: $h^\ast = \arg\max_{h \in \mathcal{H}} \epsilon_{\mathcal{D}}(h)$. For standard classification tasks, the error function is simply the indicator function of a mistake $\mathds{1}\{h(X) \not= Y\}$, and so the risk is simply the probability of a mistake $\epsilon(h) = \mathbf{E}_{(X,Y)\thicksim \mathcal{D}}[\mathds{1}\{h(X) \not= Y\}] = \mathbf{Pr}\{h(X) \not= Y\}$. For continuous response functions and multiclass classification where order matters, there are a wide choice of more complex error functions.

Of course, in practice $\mathcal{D}$ is unknown and so it is impossible to directly minimize the risk. Instead, the learner is provided with ``supervision'' in the form of a finite sample of observation-response pairs, i.e., a labeled \textit{training} data set $\mathcal{S} = \{ X_i, Y_i \}_{i=1, \dots, n}$ where $|\mathcal{S}| = n$. The learner can then approximate $\mathcal{D}$ using $\mathcal{S}$ and minimize the empirical error over $\mathcal{S}$:
\[
\hat{\epsilon}_{\mathcal{S}}(h) = \mathbf{E}_{(X,Y) \in \mathcal{S}}[\mathrm{err}(h(X), Y)] = \frac{1}{n} \sum_{i=1}^n \mathrm{err}(h(X_i), Y_i)
\]

\noindent Note that this definition of empirical risk assumes that samples $(X, Y)$ are identically independently distributed (IID), a fairly common assumption in supervised machine learning. In the \textit{empirical risk minimization} (ERM) paradigm, the learner assumes that the sample $\mathcal{S}$ is sufficiently representative of $\mathcal{D}$ such that choosing $\hat{h} = \arg\max_{h \in \mathcal{H}} \hat{\epsilon}_{\mathcal{S}}$ will yield a hypothesis $\hat{h}$ that will also have a relatively low risk $\epsilon_{\mathcal{D}}(\hat{h})$. A well known theoretical result for classification that comes from Vapnik tells us if we want to learn a ``good'' classifier from a hypothesis class $\mathcal{H}$, then we need roughly $|\mathcal{S}| = \widetilde{O}\left(d/\varepsilon^2 \log (1/\delta)\right)$ points in our training sample. Here $\varepsilon$ is the maximum deviation that we will tolerate between the true risks of $\hat{h}$ and optimal $h^\ast$ and $\delta$ is the probability with which we are willing to let this happen (i.e., we want $|\epsilon(\hat{h}) - \epsilon(h^\ast)| \leq \varepsilon$ to hold with probability $1-\delta$). Informally, $d$ represents the ``size'' of our hypothesis class; formally, it is the \textit{VC dimension}. A useful rule of thumb is that for most useful hypothesis classes, the VC dimension scales linearly with the number of parameters and so the number of training samples needed scales linearly with ``complexity'' of the model.

It is important to distinguish two cases of supervised learning, based on realizability. When the problem is \textit{realizable}, then there exists some hypothesis $h \in \mathcal{H}$ that can perfectly predict the response for every point (i.e., $\mathrm{err}(h^\ast) = 0$); in binary classification, this corresponds to the problem being ``separable'' by a hypothesis in $\mathcal{H}$. When $\mathrm{err}(h) > 0$, the problem is not realizable. The presence of \textit{label noise}, where the same point may receive different responses, further complicates this picture. If a training sample $\mathcal{S}$ contains noisy labels (perhaps due to error), this may mislead the ERM. If the true data distribution allows points to have different labels (i.e., our true labeling function is stochastic), then at best we may only be able to model $P(Y|X)$, rather than make perfect predictions.

\subsection{Greedy active learning}

\textit{Active learning} is a variation of the supervised learning paradigm where the learner does not receive access to a fully labeled data sample $\mathcal{S}$ upfront. Rather it has access to an unlabeled data sample $\mathcal{U} = \{(X, ?)\}$, as well as an \textit{oracle} that the learner can \textit{query} for the response (or label) of an observation, $Y = \mathrm{or}(X)$. The active learner is given agency to choose which individual samples to label, but each query has a cost $c$ and the learner has only a limited \textit{budget} to spend on labeling data. Thus, the active learner has dual goals that generalize the goal of the ``passive'' supervised learner: to choose simultaneously a labeled subset of observations $\mathcal{L} \subseteq \mathcal{U}$ and a hypothesis $\bar{h} = \arg\min_{h \in \mathcal{H}} \hat{\epsilon}_{\mathcal{L}}(h)$ (i.e., $\bar{h}$ is the ERM for $\mathcal{L}$) that will yield the best possible predictive performance (i.e., lowest risk $\epsilon_{\mathcal{D}}(\bar{h})$.

When evaluating active learning algorithms, we are concerned primarily with two performance properties: the quality (in terms of risk) of the hypotheses they produce and their query efficiency. Intuitively, a good active learner will use a very small number of label queries to produce a hypothesis with very small predictive error. More formally, we are interested in (1) how the error of the hypothesis produced by an active learner that chooses labeled subsample $\mathcal{L}$ compares with that of the hypothesis that we could learn from a fully labeled sample $\mathcal{S}$ where $\mathcal{L} \subseteq \mathcal{S}$; and (2) how many label queries must be made to achieve a certain level of performance, which we call \textit{label complexity} and express in \textit{Big-Oh} notation. An ideal active learner will compete with fully supervised learning with $|\mathcal{L}| \ll |\mathcal{S}|$. More realistically, we hope to at least place an upper bound on the error of active learning that is within a constant (multiplicative or additive) factor of the error of fully labeled supervised learning.

A popular approach to active learning when the problem is binary classification and realizable is a greedy strategy known as \textit{generalized binary search} (GBS), also called the \textit{splitting method}. Let $\mathcal{S}_t$ be the set of labeled data points after $t$ queries and $\mathcal{H}_t = \{ h : \hat{\epsilon}_{\mathcal{S}_t}(h) = 0 \mbox{ and } h \in \mathcal{H}\}$, i.e., the set of hypotheses from $\mathcal{H}$ that are consistent with the labeled data in $\mathcal{S}_t$. For the next ($t+1$) label query, we want to choose the unlabeled point that provokes the greatest disagreement between hypotheses in $\mathcal{H}_t$.The maximum disagreement occurs when half of the hypotheses predict one label and the rest the other. Equivalently, this minimizes the absolute value of the sum of all predicted labels: $| \sum_{h \in \mathcal{H}_t} h(x)|$ (when using $\pm1$ labels). Following this query policy, we can cut our current hypothesis space roughly in half with each query and achieve a label complexity that is roughly $O(\log (d/\varepsilon))$ for $d$ the size (e.g., VC dimension) of the hypothesis space. Indeed, this approach yields an average case label complexity that is at most $\widetilde{O}(\log d)$ times larger than the number of queries made by the optimal policy, as shown in \textbf{Theorem \ref{thm:dasgupta}}, which combines and rephrases \textit{Claim 4} and \textit{Theorem 3} from \cite{Dasgupta:2004}.

\begin{theorem}[Dasgupta ~\cite{Dasgupta:2004}]\label{thm:dasgupta}
Suppose the optimal query policy requires $m = \Theta(\log d)$ labels in expectation for target hypotheses chosen uniformly from hypothesis class $\mathcal{H}$ of (VC) dimension $d \geq e^e \approx 16$. Then the expected number of labels needed by the greedy strategy is at least $\log^2 d / \log \log d$ and at most $4 \log^2 d$.
\end{theorem}

As \cite{Dasgupta:2004} points out, the lower bound is a bit depressing, but the fact that the upper matches within a multiplicative factor is nice.

\subsection{Greedy active learning as adaptive submodularity}
IN PROGRESS BUT FIRST, MUST SLEEP.

\subsection{Other active learning applications}
IN PROGRESS BUT FIRST, MUST SLEEP. If time/space, will get into detail. If not, will write this like a related work section.

\begin{itemize}
\item Bayesian experimental design
\item Distributed learning
\item Batch mode active learning
\end{itemize}

%
%to the theoretically
%
% An observation-response pair is sampled according some underlying (and nearly always unknown) distribution $(X, Y) \thicksim \mathcal{D}$. Oftentimes, we think of responses as deterministic, so that $Y = f(X)$. The learner's goal is to select a hypothesis that most closely 
%
%$h^\ast = \argmax
%
%
%\textit{Active learning} is an interactive sub-paradigm of supervised machine learning, in which a learning algorithm is given agency over the learning process. 

\subsection{Constraint Satisfaction: Liron}
\input{csp.tex}

\subsection{Social Influence: Xinran}
With the increasing popularity of online social network site, such as Facebook and Twitter, social networks now play a fundamental role as medium for people to share, to exchange and to obtain ideas and information. Recently understanding, modeling and utilizing the social influence has become actively pursued by researchers from different aspects, for example~\cite{GJA10,KKT03,GB12,Bharathi:2007,Borodin:2010,chen2011influence,he2012influence,Budak:2011}. It turns out that submodular functions and especially submodular function maximization plays fundamental roles in solving algorithmic question associated with social influence. In this section, we mainly focus on using submodular function maximization techniques to solve two problem related to social influence, namely the influence maximization problem~\cite{KKT03} and the Network Inference problem~\cite{GJA10}.
\subsection{Influence Maximization}
Assume that now a company wants to promote its new product among the users living in a social network. The company has limited budgets to give free samples of their new product to the users in the social network. A natural question to ask is which set of the users that the company should give the free sample to, such that the overall adoption of the new product can be maximized. The question is exactly the Influence maximization problem, namely selecting a small set of seed nodes in a social networks, such that its overall influence coverage is maximized under certain diffusion model.

Among the many diffusion models, Independent Cascade (IC) model and Linear Threshold (LT) are used widely in the study of influence maximization~\cite{KKT03}. Both IC and LT models are stochastic models characterizing how influence propagates throughout the network starting from the initial seed notes.

%In the IC model, each edge $(u,v)$ in the social network is associated with an activation probability $p_{u,v}$. When a node $u$ becomes newly activated, it has exactly one chance to activate each of his still inactive neighbor $v$ with success probability $p_{u,v}$. When multiple activation attempts occur at the same same time, the attempts are considered independent from each other. For example, node $u_1$ and $u_2$ are newly activated and are both neighbor of inactive node $v$. The probability that node $u$ is activated is $1-(1-P_{u_1,v})(1-P_{u_2,v})$. The process starts with a given initial seed set $S$ and ends when no more nodes can be activated.
%
%In the LT model, instead each edge $(u,v)$ in the social network is associated with an weight $w_{u,v}$, satisfying the property that the summation of incoming edge weight for every node is less or equal than $1$, namely $\sum_{u\in N_{in}(v)}w_{u,v}\leq1$. Moreover, each node $v$ is also associated with a threshold $\theta_v$. In LT model, a node becomes activated if the incoming weight for activated neighbors exceeds its threshold. Similar to IC model, the process starts with a given initial seed set $S$ and ends when no more nodes can be activated.

For influence maximization, the objective function $\sigma(S)$, where $S$ is the initial seed set, is the expected number of activated nodes under the diffusion model. The problem is simply to maximize $\sigma(S)$ subject to the cardinality constraint $|S|\leq k$.

It has been shown that the influence maximization problem under both IC model and LT model is NP-hard~\cite{KKT03}. However, by the following theorems the problem allows efficient approximation algorithm.
 \begin{theorem}[Kempe et al.~\cite{KKT03}]\label{Thm:KKT}
The objective function of influence maximization problem under both IC and LT model is non-negative, monotone and submodular.
\end{theorem}
By the classic greedy algorithm for monotone submodular function maximization, a $1-1/e$ approximation guarantee can be achieved. The proof of the theorem is by the fact that conic combination of submodular functions is also submodular. The objective function is a expectation and can be written as
$$
\sigma(S) = \sum_{\text{outcome }X}Prob[X]\sigma(S|X),
$$
where $X$ is any realization of the stochastic diffusion process. A reachability argument for both IC model and LT model can be used to show that $\sigma(S|X)$ is submodular under any $X$. 

Though the greedy algorithm can solve the problem approximately in polynomial time, the key step of Greedy algorithm, evaluation of the marginal gain $\sigma(S\cup\{u\})-\sigma(S)$, takes long time for large networks. Large number of papers have been published to improve the efficiency of the algorithm~\cite{Chen:2010,Leskovec:2007:COD:1281192.1281239,Goyal:2011:COG:1963192.1963217,DBLP:conf/icdm/ChenYZ10,Chen:2010:SIM:1835804.1835934} , either by lazy evaluation~\cite{Leskovec:2007:COD:1281192.1281239,Goyal:2011:COG:1963192.1963217} , or approximate evaluation of the marginal gain~\cite{Chen:2010,DBLP:conf/icdm/ChenYZ10,Chen:2010:SIM:1835804.1835934} .

A more general result on Generalized Linear Threshold model has been prove containing the results on IC and LT model in~\cite{Mossel2007}. The proof uses a sophisticated stage-wise coupling argument to show the submodularity. The idea of the proof is to add the initial seeds and propagate the influence stage by stage. The key component in the proof is the anti-sense coupling used in the last stage.

A extension to influence maximization that draws much attention recently is to solve this problem under the competitive influence. Competitive influence implies that two or multiple competitive products, or ideas are propagating simultaneously in the social network. The influence maximization problem naturally extends to maximizing one's own influence~\cite{Bharathi:2007,Borodin:2010,chen2011influence} or
minimizing the influence of the competitors~\cite{he2012influence,Budak:2011} given the choices of the initial seeds of the competitors. For example, on the maximization side,~\cite{ChenCCKLRSWWY11} study the influence maximization when a user can dislike the product and propagate bad news about it. On the minimization side, ~\cite{HSCJ2012} study the influence blocking maximization, which focuses on selecting seeds to block the propagation of rumors. Both approachs solve the optimization problem by showing the objective function is monotone and submodular. The proof technique is similar to that in~\cite{KKT03}. However, the argument are much more complicated due to the interaction of competitive diffusion.
\subsection{Network Inference}
The social network structure with the strength of influence on each edge is the input to the influence maximization problem. However, in most cases, the underlying network that enables the diffusion is hidden (e.g. networks on who influenced whom). The most common observations of information diffusion are only the activation time for individual in social network (e.g. the time stamps  when a person posted a blog containing certain information or retweeted others' tweets, or buying a certain product in viral marketing application). Network inference problem focuses on revealing the diffusion network from the observed cascades occurring among the individuals in the social network. Existing approach to this problem solves a maximum likelihood estimation problem with respect to the network structure under certain diffusion models~\cite{GJA10,GB12,GDB11,SJ10}. It turns out that the likelihood function of this problem can be approximated with a submodualr function. Thus submodular function maximization can be used to solve this problem~\cite{GJA10,GB12}.

The extended IC model is used as the diffusion model in~\cite{GJA10,GB12}. In the extended IC model, each edge is associated with an activation probability. Moreover, each activation has time delay. For example, if node $v$ is activated at time $t_v$ and the activation attempt to $v$'s neighbour $u$ succeeds. Then $u$ with become activated at time $t_u+\Delta t$, where $\Delta t$ is the time delay. In the model, the delay time satisfies exponential distribution or power law distribution, namely
$$
P_d(\Delta t)\propto e^{-\frac{\Delta t}{\alpha}} \text{\ or\ } P_d(\Delta t)\propto\frac{1}{\Delta t^\alpha}
$$

Then according to the model, if $v$ which is activated at time $t_v$ and $v$ succeeds in activating node $u$ which becomes activated at time $t_u$. Then the probability this activation occurs is:
$$
P_c(v,u)=P_d(t_u-t_v)p_{vu}
$$
Also the model assumes that influence can only propagate forward in time, which means $
P_c(v,u)=0 \text{\ if\ }\ t_v>t_u.
$
Then if the pattern of cascade $c$ forms a tree $T$, the probability that the cascade is observed given the tree is
$$
P(c|T)=\prod_{(i,j)\in T}P_c(i,j)
$$
In addition, if we assume the who-infect-who relation forms a tree pattern (one is only activated by one person). Then given a certain $G$, the probability we observe the cascade $c$ would be
$$
P(c|G)=\sum_{T\in T(G)}P(c|T)P(T|G)\propto \sum_{T\in T(G)}\prod_{(i,j)\in T}P_c(i,j)
$$
where $T(G)$ is all directed spanning tree on $G$.
Therefore, if we have observed a set of cascades $C=\{c_1,c_2,\ldots\}$,
The probability of observing all these cascades is
$$
P(C|G)=\prod_{c\in C}P(c|G).
$$
Under this configuration, the network inference problem is to find an graph $\hat{G}=(V,\hat{E})$ with less than $k$ edges such that
$$
\hat{G} = \arg\max_{|E|\leq k} P(C|G)
$$
In the objective function, we sum over all spanning trees of the graph $G$, which is super-exponential. In order to make this computation feasible, only the spanning tree with the maximal likelihood is considered as a approximation, namely
$$
P(C|G)=\prod_{c\in C}\max_{T\in T(G)}P(c|T)=\prod_{c\in C}\max_{T\in T(G)}\prod_{(i,j)\in T}P_c(i,j)
$$
Then we define $F_c(G)$ as the difference between the log likelihood of cascade $c$ over graph $G$ and empty graph $\bar{K}$.
$$
F_c(G)=\max_{T\in T(G)}\log P(c|T)-\max_{T \in T(\bar{K})}\log P(c|T)
$$
and take sum over all the cascades, we have
$$
F_C(G)=\sum_{c\in C}F_c(G)
$$
$\bar{K}$ is a graph with all the nodes in $G$ and also a extra node $m$. The only edges in $\bar{K}$ are the edges from $m$ to every other nodes with activation probability $\varepsilon$ and delay $0$. The extra node represents the external influence.
Then the optimization problem can be rewritten as
\begin{equation}\label{Equ:SubObj}
G^*=\arg\max_{|E|\leq k}F_C(G)
\end{equation}
For this objective function, ~\cite{GJA10} has proved that it is monotone and submodular. Therefore, greedy algorithm can achieve $1-\frac{1}{e}$ approximation for solving this problem. The algorithm is later improve by MultiTree algorithm in ~\cite{GB12}, where the the matrix tree theorem is used to calculated the exact summation over all possible spanning trees, instead of approximating with the maximum spanning tree.

In a on-going project by Xinran He with Prof. Yan Liu, we are experimenting with using Maximum a Posteriori inference to solve the network inference problem. The previous approaches assumes no prior knowledge about the structure of the inferred graph. However, is has been well-known that social network has many unique properties, such as heavy-tail degree distribution, small diameter and community structure and so on. We propose to introduce a social network generation model over the diffusion network as a prior to incorporate the prior knowledge about network structure. Our current choice is the Kronecker graphs model~\cite{JDJCZ10,KJ11}. The Kronecker graphs model is a parametric model which can provide a probability for the existence of each edge in the social network. The existence of each edge is considered independent under this model. Using this model as a prior, we can change the objective function $F_C(G)$ in Equation~\ref{Equ:SubObj} to
$$
F'_C(G) = F_C(G) + \sum_{e\in E}(\log Prob[e\ \text{exists}]-\log Prob[e\ \text{not exists}]).
$$
 After adding a modular function to a submodular function, the resulted $F'_C(G)$ is still submodular, however it may not necessary be monotone any more. As a result, the simple greedy algorithm can not be used to solve this problem. Instead, we can use the algorithm proposed in~\cite{FeldmanNS11} with a $1/2+o(1)$ approximation guarantee.
%\bibliographystyle{abbrv}
%\bibliography{ref}



% \subsection{OLD STUFF}
% \[
%  A(s^t) = \underset{a}{\operatorname{argmax }}\text{ } r_a^t + \mathbb{E} V^{S^{t+1}}(K(s^t,a))
% \]
% where, $s \text{ is the current state }\\
% \mathbb{E} V^{S^{t+1}} \text{ is the expected value over all future states }\\
% K(s^t,a) \text{ is the transition function which takes state, $s^t$ and action $a$ as input }\\$
% 
% In cases where both the transition probabilities or the reward distributions are unknown, one can appeal to Reinforcement Learning methods. In such methods the system needs to execute a series of actions and observe rewards to approxmiate the reward function. In Interactive perception problems, both the transition model and the reward functions are hard to model a priori. Moreover given that the state space of these problems are incredibly large, the probability of being able to explore the entire state space is minimal. In order to circumvent these issues we can try to approximate the reward function associated with each action and be agnostic to the states in which these actions are executed. The class of problems where we approximate the reward distribution for each action while simultaneously optimizing decisions based on existing knowledge are commonly referred to as armed bandit problems in the literature. 
% 
% These problems efficiently balance the exploration/exploitation tradeoff. Here the observed rewards are used to approximate the reward distributions for each action. Hence each action can be thought of as being analgous to making a measurement (i.e sampling the reward distribution). In classic bandit approaches, actions are selected via a greedy policy with some notion exploration/exploitation tradeoff encoded in the action selection policy. For us to ensure that the action being selected for such a strategy is optimal we need to ensure that the action (or measurement) we take is optimal if it were the last measurement we were allowed to make.
% 
% From our previous action selection definition used for MDPs we can re-write the expected value over the next state which balances exploration and exploitation for a multi-armed bandit, as follows
% \[
%  \mathbb{E} V^{S^{t+1}}(K(s^t,a)) = (\tau - t)(\underset{a'}{\operatorname{max }}\text{ } r_{a'}^t) + (\tau - t) \nu_{a'}^{KG,t}
% \]
% 
% where, $\nu_{a'}^{KG,t} \text{ is the knowledge gradient }$\\
% 
% The knowledge gradient can be thought of the direction(gradient) along which we move in the action(policy) space in order to maximize our objective function, i.e expected reward. This definition will become clearer in the following sections. In the literature, the knowledge gradient is defined as ``choosing the measurement that would be optimal if it were the last measurement we were allowed to make'' \cite{KG_Ryzhov_12}, hence it is the gradient that would reduce the variance in the expected reward distribution.\\
% \[
% \nu_{a'}^{KG,t} = \mathbb{E}^t[(\underset{a'}{\operatorname{max }}\text{ } r_{a'}^{t+1}) - (\underset{a'}{\operatorname{max }}\text{ }  r_{a'}^{t}) ] 
% \]
% $\nu_{a'}^{KG,t}$ can be considered as the marginal value of information. Hence this summarizes our action selection strategy for both the finite and infinite horizon problem as follows:\\
% \[
%  \text{ {\bf Finite Horizon: }}A(s^t) = \underset{a}{\operatorname{argmax }}\text{ } r_a^t + (\tau - t)\nu_{a'}^{KG,t}
%  \]
%  \[
%  \text{ {\bf Infinite Horizon: }}A(s^t) = \underset{a}{\operatorname{argmax }}\text{ } r_a^t + \frac{\gamma}{1-\gamma}\nu_{a'}^{KG,t}
% \]
% where $\gamma$ is the discount factor for the infinite horizon case. When relating to the context of classic multiarm bandits, instead of simply picking the action that looks best, we pick the action with an added uncertainity parameter that balances exploration and exploitation.
% \begin{center}
%  \begin{align*}
%  p(\theta | D) &\propto p(D| \theta)p(\theta)  \\
%  &\propto \text{Ber}(\theta)\text{Be}(\alpha,\beta)\\
%  &\propto \text{Ber}(N_1|\theta)\text{Be}(\alpha,\beta)\\
%   &\propto \text{Be}(\alpha + N_1, \beta + N_0)
% \end{align*}
% \end{center}
% 
% This simplifies our updates to simply adding the number of observed events $N$ to the hyperparameters
% 
% \begin{center}
%  \begin{align*}
%  \alpha_a^{t+1} &= \alpha_a^t + N  \\
%  \beta_a^{t+1} &= \beta_a^t + (1 - N)\\
% \end{align*}
% \end{center}
% 

\nocite{*}
\bibliographystyle{./IEEEtran}
\bibliography{bibliography}
\end{document}
